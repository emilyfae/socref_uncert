% Template for Cogsci submission with R Markdown

% Stuff changed from original Markdown PLOS Template
\documentclass[10pt, letterpaper]{article}

\usepackage{cogsci}
\usepackage{pslatex}
\usepackage{float}
\usepackage{caption}

% amsmath package, useful for mathematical formulas
\usepackage{amsmath}

% amssymb package, useful for mathematical symbols
\usepackage{amssymb}

% hyperref package, useful for hyperlinks
\usepackage{hyperref}

% graphicx package, useful for including eps and pdf graphics
% include graphics with the command \includegraphics
\usepackage{graphicx}

% Sweave(-like)
\usepackage{fancyvrb}
\DefineVerbatimEnvironment{Sinput}{Verbatim}{fontshape=sl}
\DefineVerbatimEnvironment{Soutput}{Verbatim}{}
\DefineVerbatimEnvironment{Scode}{Verbatim}{fontshape=sl}
\newenvironment{Schunk}{}{}
\DefineVerbatimEnvironment{Code}{Verbatim}{}
\DefineVerbatimEnvironment{CodeInput}{Verbatim}{fontshape=sl}
\DefineVerbatimEnvironment{CodeOutput}{Verbatim}{}
\newenvironment{CodeChunk}{}{}

% cite package, to clean up citations in the main text. Do not remove.
\usepackage{cite}

\usepackage{color}

% Use doublespacing - comment out for single spacing
%\usepackage{setspace}
%\doublespacing


% % Text layout
% \topmargin 0.0cm
% \oddsidemargin 0.5cm
% \evensidemargin 0.5cm
% \textwidth 16cm
% \textheight 21cm

\title{Children's social referencing reflects sensitivity to graded uncertainty}


\author{{\large \bf Emily  Hembacher} \\ \texttt{ehembach@stanford.edu} \\ Department of Psychology \\ Stanford University \And {\large \bf Benjamin deMayo} \\ \texttt{bedemayo@stanford.edu} \\ Department of Psychology \\ Stanford University \And {\large \bf Michael C. Frank} \\ \texttt{mcfrank@stanford.edu} \\ Department of Psychology \\ Stanford University}

\begin{document}

\maketitle

\begin{abstract}
The ability to monitor epistemic uncertainty is critical for
self-directed learning. However, we still know little about young
children's ability to detect uncertainty in their mental
representations. Here we asked whether a spontaneous information
gathering behavior -- social referencing -- is driven by uncertainty
during early childhood. Children ages 2-5 completed a word-learning task
in which they were presented with one or two objects, heard a label, and
were asked to put the labeled object in a bucket. Referential ambiguity
was manipulated through the number of objects present and their
familiarity. In Experiment 1, when there were two novel objects and a
novel label, the referent was ambiguous; when there were two familiar
objects, or only one novel or familiar object, the referent was known or
could be inferred. In Experiment 2, there were either two novel objects,
two familiar objects, or one familiar and one novel object; in the
latter case the referent could be inferred by excluding the familiar
object. To further manipulate the availability of referential cues, the
experimenter gazed at either the target or the center of the table while
labeling the object. In both experiments, children looked at the
experimenter more often while making their response when the referent
was ambiguous. In Experiment 2, children also looked at the experimenter
more when there was one familiar and one novel object, but only when the
experimenter's gaze during labeling was uninformative. These results
suggest that children's social referencing is a sensitive index of
graded epistemic uncertainty.

\textbf{Keywords:}
social referencing; help seeking; word learning; uncertainty.
\end{abstract}

Preschoolers quickly learn new concepts, rules, and language. They also
actively explore and ask questions in ways that seem targeted to
maximize learning (Chouinard, Harris, \& Maratsos, 2007; Schulz \&
Bonawitz, 2007). However, we still have an incomplete understanding of
young children's ability to monitor their own mental states, in
particular, their epistemic uncertainty (Schneider, 2008; Sodian,
Thoermer, Kristen, \& Perst, 2012). Do preschool-aged children monitor
uncertainty and guide their learning behaviors on the basis of this
monitoring, or is early learning better characterized as a process of
integrating information that is largely generated externally, for
example, by social partners who act as teachers (Csibra \& Gergely,
2006)?

A hallmark of successful uncertainty monitoring is being less confident
when the probability of accuracy is lower (Robinson, Johnson, \&
Herndon, 1997). This includes awareness of complete ignorance, but also
of graded evidence in mental representations (Ghetti, Lyons, Lazzarin,
\& Cornoldi, 2008), which may be important for predicting outcomes and
seeking disambiguating information. There is mixed evidence about
whether young children can accomplish this. For example, preschool-aged
children fail at reporting on their ongoing thoughts (J. H. Flavell,
Green, Flavell, Harris, \& Astington, 1995), and 3-year-olds report
being equally confident about correct and incorrect responses in a
memory task (Hembacher \& Ghetti, 2014). However, other studies have
shown that preschoolers can distinguish between their own correct and
incorrect answers in other tasks, although they are typically
overconfident (Coughlin, Hembacher, Lyons, \& Ghetti, 2015; Lyons \&
Ghetti, 2013; Paulus, Proust, \& Sodian, 2013).

There is also protracted development throughout childhood of other
metacognitive abilities. For example, children gradually improve in
their ability to estimate future performance (Destan, Hembacher, Ghetti,
\& Roebers, 2014; Lipowski, Merriman, \& Dunlosky, 2013) and selectively
allocate learning time to difficult or less-well-learned materials
(Metcalfe \& Finn, 2013). However, most of these studies have relied
upon explicit reports of uncertainty or learning progress, for example,
by asking children to use a confidence scale. Perhaps children learn to
respond appropriately to uncertainty in everyday learning situations
before they can bring it fully into consciousness and report on it.

Several studies have provided evidence that children's spontaneous
information-seeking behaviors might track uncertainty. Call and
Carpenter (2001) had 2-year-olds choose between several tubes to find a
hidden sticker. They found that the toddlers were more likely to peek
inside a tube before choosing when they had not seen the baiting of the
tubes compared to when they had, suggesting they were aware of their
ignorance and managed to delay their response until they were
sufficiently confident. In another study, Goupil, Romand-Monnier, \&
Kouider (2016) found that 20-month-olds were more likely to seek help by
looking at their parents when they were unable to respond accurately in
a memory task. Thus, spontaneous information-gathering behaviors may
provide a window into early uncertainty monitoring, and allow us to ask
questions about it's development.

Here, we focus on the role of uncertainty in guiding social referencing
-- one form of information gathering -- during word learning.
Referencing a social partner can provide several forms of disambiguating
information. First, children can follow a speaker's gaze direction to
infer the referent of a new word, as people tend to look at objects they
are referring to. By the second year of life infants follow a speaker's
gaze and map labels to objects on the basis of gaze direction (D. A.
Baldwin, 1991). There is also evidence that infants' propensity for
gaze-following predicts later language development (Carpenter, Nagel,
Tomasello, Butterworth, \& Moore, 1998), highlighting the importance of
this behavior for learning. In addition to monitoring gaze direction,
people may reference a social partner's emotional reaction to a stimulus
or event, which can help disambiguate the appropriateness of a response
{[}ref{]}. Finally, looking at a social partner can be taken as a bid
for help {[}ref{]}, and may result in explicit instruction.

Social referencing can be an efficient source of disambiguating
information, but is it driven by uncertainty during early childhood? It
could be that social referencing is relatively cheap in terms of effort
and time, so selectivity may be unnecessary. Similarly, children may
simply respond to external events, for example, by fixating towards
someone who is speaking, without monitoring their own knowledge states.
Vaish, Demir and Baldwin (2011) addressed this question with infants.
Thirteen to 18-month-olds sat across from an experimenter who produced a
label (e.g., ``a modi!) in the presence of one or two objects. They
found that infants looked up to the experimenter more often when there
were two objects present, and the referent was thus ambiguous. This
finding provides initial evidence that infants selectively reference a
speaker when they are uncertain about the referent of a label, rather
than indiscriminantly looking at someone who is speaking. In the present
research, we capitalize on this method to ask whether preschool-aged
children's social referencing is selective based on uncertainty
generated by graded evidence.

\begin{CodeChunk}
\captionsetup{width=0.8\columnwidth}\begin{figure}[h]

{\centering \includegraphics{figs/design-1} 

}

\caption[Study design for Experiments 1 and 2]{Study design for Experiments 1 and 2.}\label{fig:design}
\end{figure}
\end{CodeChunk}

\section{Experiment 1}\label{experiment-1}

The present research adapts the method of Vaish et al. to examine
whether preschool-aged children's social referencing is sensitive to
uncertainty based on referential ambiguity, and whether it reflects
graded uncertainty about a word-object mapping. In Experiment 1, we
examined whether children would reference a speaker more often when she
produced a referentially ambiguous label. Children sat across from an
experimenter who labeled an object on the table between them (Figure
\ref{fig:design}). Across trials, there were either one or two objects
on the table, which were either familiar or unfamiliar to the child. We
expected children to be uncertain about the label's referent only on
trials with two unfamiliar objects, when the object-label mapping was
not known and could not be inferred. After labeling an object, the
experimenter asked the child to place the named object in a bucket.

We were interested in the amount of social referencing children
exhibited across the trial. We considered four different phases of each
trial based on the notion that children might expect different social
information at different stages of the task. Specifically, we predicted
that children might expect the speaker's gaze direction to be
informative during the labeling itself, as speakers tend to look at
objects they refer to. We predicted that later in the trial, as children
reached for an object and placed it in the bucket, children might expect
confirmation of the accuracy of their choice. We were interested in
whether children would reference the speaker to a greater degree across
these phases when referential ambiguity was high, which would indicate
sensitivity to epistemic uncertainty.

\subsection{Methods}\label{methods}

\subsubsection{Participants}\label{participants}

We recruited a planned sample of 80 children ages 2-5 years from the
Children's Discovery Museum in San Jose, California\footnote{Planned
  sample size, exclusion criteria, and analysis plan preregistered at
  \url{https://osf.io/y7mvt}}. The sample included 20 2-year-olds (mean
age 31.71 months), 20 3-year-olds (mean age 42.65 months), 20
4-year-olds (mean age 55.85 months), and 20 5-year-olds (mean age 65.11
months). An additional 20 children participated but were removed from
analyses because they heard English less than 75\% of the time at home
(\emph{n} = 10), because they were unable to complete at least half of
the trials in the task (\emph{n} = 4), because of parental interference
(\emph{n} = 1), or due to experimenter or technical errors (\emph{n} =
5).

\subsubsection{Stimuli and Design}\label{stimuli-and-design}

Children were presented with one or two objects, heard a label, and were
asked to put the labeled object in a bucket. Half of the objects were
selected to be familiar to children (e.g., a cow) and half were selected
to be novel (e.g., a nozzle). There were four trial types: one-familiar,
one-novel, two-familiar, and two-novel. There were three trials of each
type, for a total of twelve trials. Trial types were presented
sequentially in an order that was counterbalanced across participants.
The assignment of individual objects to trial types was counterbalanced.
On familiar trials, the familiar label for the target object was used
(e.g., ``cow''). On novel trials, a novel label was used (e.g.,
``dawnoo'').

The critical manipulation was of referential ambiguity; on one-familiar
and two-familiar trials, there was no referential ambiguity, as children
were expected to be certain about the objects and their labels.
Similarly, on one-novel trials, children were expected to be certain
about the label referent as there was only one option. However, on
two-novel trials, the referent was ambiguous, as the novel label could
apply to either novel object.

\subsubsection{Procedure}\label{procedure}

Throughout the study, the child sat at one end of a large circular
table, and the experimenter stood at the opposite end. Each trial of the
task proceeded as follows: the experimenter placed one or two objects on
the left and/or right sides of the table, out of reach of the child so
that the child could not interact with the toys during the labeling
event. For one-object trials, the location of the object (left or right)
alternated between trials. After placing the objects, the experimenter
said ``Hey look, there's a (target) here.'' The experimenter gazed at
the center of the table rather than the object she was labeling because
we wanted to preserve the referential ambiguity throughout the trial.
The experimenter waited approximately two seconds (based on a visual
metronome placed within view) before saying, ``Can you put the (target)
in the bucket?'' She then pushed the object(s) forward within reach of
the child, and placed a plastic bucket in the center of the table, also
within reach of the child. Prior to the twelve experimental trials,
there were two training trials: a one-familiar trial and a two-familiar
trial, to acquaint the child with the procedure. A camera placed to the
side of the experimenter captured the participant's face, so that
looking behavior could be coded from video.

\subsubsection{Coding procedure}\label{coding-procedure}

Videos were coded using DataVyu software (\url{http://datavyu.org}).
First, each trial was divided into four temporal phases: a \emph{label}
phase, which began at the utterance of the target label and ended when
the experimenter began to slide the objects, a \emph{slide} phase, which
encompassed the sliding of the objects into the child's reach, a
\emph{planning phase}, which began at the end of the slide and ended
when the child touched an object, and a \emph{response} phase, which
began when the child touched an object and ended when the child released
the object into the bucket. After onsets and offsets of these phases had
been coded, the coder recorded the number of looks the child made to the
experimenter during each phase. We opted to code the number of looks
rather than the duration of looks because we felt that looks from the
stimuli to the experimenter and vice versa might allow children to
integrate social and nonsocial information to solve the problem of
reference. A second coder coded the number of looks for a quarter of the
trials for each participant to establish reliability. For Experiment 1,
\#\#\% of trials were given the same number of looks by both coders. For
Experiment 2, \#\#\% of trials were given the same number of looks by
both coders.

\subsection{Results and Discussion}\label{results-and-discussion}

\begin{CodeChunk}
\begin{figure*}[h]

{\centering \includegraphics[width=6.5in,height=4in]{figs/results_e1-1} 

}

\caption[Results of Experiment 1]{Results of Experiment 1. Number of looks to the experimenter across phases and conditions. Error bars are 95 percent confidence intervals.}\label{fig:results_e1}
\end{figure*}
\end{CodeChunk}

Results of Experiment 1 are presented in Figure \ref{fig:results_e1}. To
test our prediction that referential ambiguity (i.e., having two novel
objects) would produce more social referencing, we fit mixed-effects
linear regression models separately for each phase with the following
structure:
\texttt{number\ of\ looks\ \textasciitilde{}\ number\ of\ objects\ *\ familiarity\ *\ age\ in\ months\ +\ (number\ of\ objects\ +\ familiarity\ \textbar{}\ SID)}.
A single model with phase as a factor did not converge.

We did not find any main or interaction effects of number of objects,
familiarity, or age on number of looks during the label or slide phases.
However, we found an interaction effect of number of objects and
familiarity during the planning (\(\beta\) = 0.21, \emph{p} \textless{}
.001) and response phases (\(\beta\) = 0.6, \emph{p} \textless{} .001),
such that 2-novel trials were associated with more looking. There was no
interaction with age in either phase\footnote{\url{https://github.com/emilyfae/socref_uncert}}.
In summary, children looked to the experimenter more often when planning
and executing a response under uncertainty. These results suggest that
children were aware that they did not have sufficient knowledge to
answer independently, and they attempted to resolve their uncertainty
using social referencing.

We did not find the expected effect of referential ambiguity in the
label phase. It is possible that children failed to predict that they
would need more information until later in the trial, when they were
actually faced with making a decision. Another possibility is that
children's looking was at ceiling during the labeling phase, perhaps
because children look at someone who is speaking regardless of the need
for referential disambiguation. A third possibility is that this is an
artifact of our design, in which the experimenter gazed at the center of
the table rather than the referent of her label. Children may have
realized that the experimenter's gaze direction during labeling was not
informative. Experiment 2 tests this possibility and examines whether
children's social referencing is sensitive to graded uncertainty.

\section{Experiment 2}\label{experiment-2}

Experiment 2 was designed to replicate Experiment 1 and to investigate
whether children's social referencing is sensitive to uncertainty based
on graded evidence about a label's referent. Since we did not observe
any difference between the one-familiar and one-novel trials, we
eliminated single-object trials, leaving the 2-familiar and 2-novel
trials. In addition, we added 1-novel-1-familiar trials (``mutual
exclusivity trials''). For these trials, we expected that children would
be able to infer the referent by excluding the familiar object as a
possibility(Markman \& Wachtel, 1988). We predicted that children might
be less certain about their choice on these trials compared to when the
label and referent were familiar to them (familiar trials), but more
confident than when there are no cues to reference (novel trials).

In addition, we manipulated between participants whether or not the
experimenter's gaze during labeling was informative (she gazed at either
the referent of her label or the center of the table), allowing us to
determine whether children selectively reference gaze during labeling
when gaze is expected to be informative. The manipulation of
informativity of gaze during labeling also meant that participants in
the referential gaze condition had an additional referential cue, which
might decrease uncertainty in the remainder of the trial.

In Experiment 1, we did not observe an effect of age on looking. Thus,
we restricted the current sample to 3- and 4-year-olds.

\subsection{Methods}\label{methods-1}

\subsubsection{Participants}\label{participants-1}

We recruited a planned sample of 80 children ages 3-4 years from the
Children's Discovery Museum in San Jose,
California\footnote{Planned sample size, exclusion criteria, and analysis plan preregistered at https://osf.io/y7mvt/}.
The sample included 40 3-year-olds (mean age 42.89 months) and 40
4-year-olds (mean age 53.47 months). An additional 20 children
participated but were removed from analyses because they heard English
less than 75\% of the time at home (\emph{n} = 9), because they were
unable to complete at least half of the trials in the task (\emph{n} =
7), or due to experimenter or technical errors (\emph{n} = 4).

\subsubsection{Stimuli and Design}\label{stimuli-and-design-1}

The stimuli and design were similar to Experiment 1, except that we
eliminated 1-object trials. Instead, we included three trial types:
2-familiar (``familiar''), 2-novel (``novel''), and 1-novel-1-familiar
(``mutual exclusivity''). There were four of each trial type, totaling
twelve trials. In addition, we manipulated the experimenter's gaze
behavior between participants. For half of the participants, she looked
at the center of the table on every trial; for the remaining half, she
looked at the object she referred to on every trial.

\subsubsection{Procedure}\label{procedure-1}

The procedure was identical to Experiment 1, except that there were
three practice trials rather than two, so that children could experience
every trial type.

\subsection{Results and Discussion}\label{results-and-discussion-1}

\begin{CodeChunk}
\begin{figure*}[h]

{\centering \includegraphics[width=6.5in,height=3.25in]{figs/results_e2-1} 

}

\caption[Results of Experiment 2]{Results of Experiment 2. Number of looks to the experimenter across phases and conditions. Error bars are 95 percent confidence intervals.}\label{fig:results_e2}
\end{figure*}
\end{CodeChunk}

Results of Experiment 2 are presented in Figure \ref{fig:results_e2}. To
quantify the main and interactive effects of familiarity, gaze
informativity, phase, and age on social referencing, we fit a
mixed-effects linear regression model with the following structure:
\texttt{number\ of\ looks\ \textasciitilde{}\ familiarity\ *\ age\ in\ months\ *\ gaze\ *\ phase\ +\ (familiarity\ \textbar{}\ SID)}.
In contrast to Experiment 1, a model with phase as a predictor
converged.

We were interested in several questions. First, do children reference a
speaker more often when the objects and label are novel? Phase
interacted with familiarity such that the response phase of novel trials
was associated with significantly more looks (\(\beta\) = 0.51, \emph{p}
\textless{} .001). This is consistent with our finding from the linear
model of the response phase in Experiment 1. However, in contrast to
Experiment 1, we did not observe that looking was significantly greater
for novel trials in the planning phase.

We were also interested in whether mutual exclusivity trials would
elicit an intermediate amount of social referencing. We observed a
three-way interaction of familiarity, gaze, and phase, such that the
response phase of mutual exclusivity trials in the no-referential-gaze
condition was associated with significantly more looks (\(\beta\) =
0.39, \emph{p} \textless{} .01). In sum, mutual exclusivity trials were
associated with greater looking during the response phase when the
experimenter provided informative gaze compared to when she did not.
This finding is intriguing given that children should be able to solve
mutual exclusivity trials without gaze information. Instead, they remain
relatively uncertain while executing a decision if excluding the
familiar object is their only cue to reference, but this uncertainty is
resolved if the experimenter gazes at the correct object during
labeling. On the other hand, informative gaze during labeling did not
lessen the amount of social referencing during the response phase for
novel trials, suggesting that gaze information alone was not sufficient
to reduce uncertainty. Children may require multiple cues

This provides evidence for sensitivity to graded evidence.

Finally, we observed a four-way interaction such that the response phase
of novel trials in the gaze condition was associated with more looking
with increasing age (\(\beta\) = 0.06, \emph{p} \textless{} .01),
suggesting that children may become more selective in their social
referencing as they get older. It may be that children improve in their
ability to monitor the need for disambiguating information, or they may
become more likely to recognize that social information can be a source
of disambiguation.

Lastly, we again did not find selective social referencing during the
label phase, even when referential gaze was available. This rules out
the possibility that children were less selective during this phase
because they learned that gaze direction was not informative.

\section{General Discussion}\label{general-discussion}

Being able to detect uncertainty in mental representations seems
critical for self-guided learning. During the preschool years, children
are increasingly able to actively gather their own information through
help-seeking and exploration. Do children monitor their own uncertainty
to guide these behaviors, or are external features of the environment
sufficient to guide learning (e.g., children might ask for help with a
complex looking toy in response to perceptual features rather than
epistemic states). Here, we examined whether children's social
referencing in a word-learning scenario was calibrated to uncertainty
about a label's referent.

We found strong evidence of selective social referencing under
referential ambiguity when children were making their decision about
which object was the target. In Experiment 1, we additionally found
evidence for selective social referencing as children planned their
decision, though this was not replicated in Experiment 2. We speculate
that children may have referenced the speaker during the decision
process because they expected confirmation of the accuracy of their
choice, either implicitly through the adult's facial expressions, or
through explicit feedback.

Importantly, we also found evidence for selectivity in social
referencing based on graded uncertainty. In Experiment 2, we manipulated
the amount of evidence available to children in two ways. First, we
included mutual exclusivity trials consisting of a novel label paired
with one familiar and one novel object, which we predicted might lead to
intermediate levels of uncertainty compared to two-familiar and
two-novel trials. Second, we manipulated whether or not the speaker's
gaze direction was informative. We found that children treated mutual
exclusivity trials more like familiar trials when they had received
helpful gaze, but more like novel trials when they had not, suggesting
that the combination of mutual exclusivity cues and gaze direction cues
were required for children to feel confident about the object-label
mapping. It is possible that they remain uncertain about a new label
even after they have acquired it, if they have only heard it once and
not received confirmation of its accuracy, for example, through gaze
monitoring. Importantly, informative gaze during labeling did not lessen
the amount of social referencing during the response phase for novel
trials, suggesting that gaze information alone was not sufficient to
reduce uncertainty.

On the other hand, we found no evidence for selectivity as the object
was being labeled, or as the objects were being slid into reach. One
interpretation of this pattern of results is that preschool-aged
children do not recognize the need for disambiguating information when a
referent is ambiguous until they are in the position of needing to make
a decision. However, another possibility is that children spontaneously
look at a speaker regardless of the need for disambiguating information,
and additional looking on top of this baseline was not needed or
possible. A related possibility is that the labeling phase of our task
was too short for children to produce extra looking on top of baseline
in response to uncertainty. A follow-up to the present work that
includes a longer labeling period or a greater reward tradeoff between
attentional options would help to distinguish among these possibilities.
Overall, these results provide evidence that preschool-aged children
monitor graded uncertainty in their mental representations and act on
that uncertainty through information-gathering behaviors.

\section{Acknowledgements}\label{acknowledgements}

We thank Veronica Cristiano for assisting with data collection.

\section{References}\label{references}

\setlength{\parindent}{-0.1in} \setlength{\leftskip}{0.125in} \noindent

\hypertarget{refs}{}
\hypertarget{ref-Baldwin1991}{}
Baldwin, D. A. (1991). Infants' contribution to the achievement of joint
reference. \emph{Child Development}, \emph{62}(5), 875--890.

\hypertarget{ref-Call2001}{}
Call, J., \& Carpenter, M. (2001). Do apes and children know what they
have seen? \emph{Animal Cognition}, \emph{3}(4), 207--220.

\hypertarget{ref-Carpenter1998}{}
Carpenter, M., Nagel, K., Tomasello, M., Butterworth, G., \& Moore, C.
(1998). Social Cognition, Joint Attention, and Communicative Competence
from 9 to 15 Months of Age
. \emph{Monographs of the Society for
Research in Child Development}, \emph{63}(4), i--iii--v--vi--1--174.

\hypertarget{ref-Chouinard2007}{}
Chouinard, M. M., Harris, P. L., \& Maratsos, M. P. (2007). Children's
questions: A mechanism for cognitive development. \emph{Monographs of
the Society for Research in Child Development}, \emph{72}, 1--129.

\hypertarget{ref-Coughlin2015}{}
Coughlin, C., Hembacher, E., Lyons, K. E., \& Ghetti, S. (2015).
Introspection on uncertainty and judicious help-seeking during the
preschool years. \emph{Developmental Science}, \emph{18}(6), 957--971.

\hypertarget{ref-Csibra2006}{}
Csibra, G., \& Gergely, G. (2006). Social learning and social cognition:
The case for pedagogy. In Y. Munakata \& M. H. Johnson (Eds.),
\emph{Processes of change in brain and cognitive development} (pp.
249--274). Oxford: Oxford University Press: Oxford University Press.

\hypertarget{ref-Destan2014}{}
Destan, N., Hembacher, E., Ghetti, S., \& Roebers, C. M. (2014). Early
metacognitive abilities: The interplay of monitoring and control
processes in 5- to 7-year-old children. \emph{Journal of Experimental
Child Psychology}, \emph{126}(C), 213--228.

\hypertarget{ref-Flavell1995}{}
Flavell, J. H., Green, F. L., Flavell, E. R., Harris, P. L., \&
Astington, J. W. (1995). Young Children's Knowledge about Thinking.
\emph{Monographs of the Society for Research in Child Development},
i--iii--v--vi--1--113.

\hypertarget{ref-Ghetti2008}{}
Ghetti, S., Lyons, K. E., Lazzarin, F., \& Cornoldi, C. (2008). The
development of metamemory monitoring during retrieval: The case of
memory strength and memory absence. \emph{Journal of Experimental Child
Psychology}, \emph{99}(3), 157--181.

\hypertarget{ref-Goupil2016}{}
Goupil, L., Romand-Monnier, M., \& Kouider, S. (2016). Infants ask for
help when they know they dont know. \emph{Proceedings of the National
Academy of Sciences}, \emph{113}(13), 3492--3496.

\hypertarget{ref-Hembacher2014}{}
Hembacher, E., \& Ghetti, S. (2014). Dont Look at My Answer Subjective
Uncertainty Underlies Preschoolers Exclusion of Their Least Accurate
Memories. \emph{Psychological Science}, \emph{25}(9), 1--9.

\hypertarget{ref-Lipowski2013}{}
Lipowski, S. L., Merriman, W. E., \& Dunlosky, J. (2013). Preschoolers
can make highly accurate judgments of learning. \emph{Developmental
Psychology}, \emph{49}(8), 1505--1516.

\hypertarget{ref-Lyons2013}{}
Lyons, K. E., \& Ghetti, S. (2013). I Don\&apos;t Want to Pick!
Introspection on Uncertainty Supports Early Strategic Behavior.
\emph{Child Development}, \emph{84}(2), 726--736.

\hypertarget{ref-Markman1988}{}
Markman, E. M., \& Wachtel, G. F. (1988). Children's use of mutual
exclusivity to constrain the meanings of words. \emph{Cognitive
Psychology}, \emph{20}, 121--157.

\hypertarget{ref-Metcalfe2013}{}
Metcalfe, J., \& Finn, B. (2013). Metacognition and control of study
choice in children. \emph{Metacognition and Learning}, \emph{8}(1),
19--46.

\hypertarget{ref-Paulus2013}{}
Paulus, M., Proust, J., \& Sodian, B. (2013). Examining implicit
metacognition in 3.5-year-old children: an eye-tracking and
pupillometric study. \emph{Frontiers in Psychology}, 1--7.

\hypertarget{ref-Robinson1997}{}
Robinson, M. D., Johnson, J. T., \& Herndon, F. (1997). Reaction Time
and Assessments of Cognitive Effort as Predictors of Eyewitness Memory
Accuracy and Confidence. \emph{Journal of Applied Psychology},
\emph{82}, 416--425.

\hypertarget{ref-Schneider2008}{}
Schneider, W. (2008). The Development of Metacognitive Knowledge in
Children and Adolescents: Major Trends and Implications for Education.
\emph{Mind, Brain, and Education}, \emph{2}(3), 1--8.

\hypertarget{ref-Schulz2007}{}
Schulz, L. E., \& Bonawitz, E. B. (2007). Serious fun: Preschoolers
engage in more exploratory play when evidence is confounded.
\emph{Developmental Psychology}, \emph{43}(4), 1045--1050.

\hypertarget{ref-Sodian2012}{}
Sodian, B., Thoermer, C., Kristen, S., \& Perst, H. (2012).
Metacognition in infants and young children. In M. J. Beran, J. Brandl,
J. Perner, \& J. Proust (Eds.), \emph{Foundations of metacognition} (pp.
119--133).

\hypertarget{ref-Vaish2011}{}
Vaish, A., Demir, Ö. E., \& Baldwin, D. (2011). Thirteen- and
18-month-old Infants Recognize When They Need Referential Information.
\emph{Social Development}, \emph{20}(3), 431--449.

\end{document}
