% Template for APA submission with R Markdown

% Stuff changed from PLOS Template
\documentclass[a4paper,man,apacite,floatsintext]{apa6}
\usepackage{apacite}

% amsmath package, useful for mathematical formulas
\usepackage{amsmath}
% amssymb package, useful for mathematical symbols
\usepackage{amssymb}

% hyperref package, useful for hyperlinks
\usepackage{hyperref}

% graphicx package, useful for including eps and pdf graphics
% include graphics with the command \includegraphics
\usepackage{graphicx}

% Sweave(-like)
\usepackage{fancyvrb}
\DefineVerbatimEnvironment{Sinput}{Verbatim}{fontshape=sl}
\DefineVerbatimEnvironment{Soutput}{Verbatim}{}
\DefineVerbatimEnvironment{Scode}{Verbatim}{fontshape=sl}
\newenvironment{Schunk}{}{}
\DefineVerbatimEnvironment{Code}{Verbatim}{}
\DefineVerbatimEnvironment{CodeInput}{Verbatim}{fontshape=sl}
\DefineVerbatimEnvironment{CodeOutput}{Verbatim}{}
\newenvironment{CodeChunk}{}{}

% cite package, to clean up citations in the main text. Do not remove.
\usepackage{cite}

\usepackage{color}

% Use doublespacing - comment out for single spacing
%\usepackage{setspace}
%\doublespacing


% Text layout
\topmargin 0.0cm
\oddsidemargin 0.5cm
\evensidemargin 0.5cm
\textwidth 16cm
\textheight 21cm

% Bold the 'Figure #' in the caption and separate it with a period
% Captions will be left justified
\usepackage[labelfont=bf,labelsep=period,justification=raggedright]{caption}


% Remove brackets from numbering in List of References
\makeatletter
\renewcommand{\@biblabel}[1]{\quad#1.}
\makeatother


% Leave date blank
\date{}

%\pagestyle{myheadings}
%% ** EDIT HERE **


%% ** EDIT HERE **
%% PLEASE INCLUDE ALL MACROS BELOW

%% END MACROS SECTION


% ALL OF THE TITLE PAGE INFORMATION IS SPECIFIED IN THE YAML
\title{\textbf{Children's social referencing reflects sensitivity to graded uncertainty}}
\shorttitle{Social referencing and uncertainty}

\author{Emily Hembacher, Benjamin deMayo, Michael C. Frank}

\affiliation{Department of Psychology, Stanford University}

\authornote{}
\abstract{This study examined a spontaneous information gathering behavior --
social referencing -- and its relation to epistemic uncertainty during
early childhood. Children ages 2-5 (\emph{n}=160) identified the
referents of familiar and novel labels. Referential ambiguity was
manipulated through the number of objects present and their familiarity
(Experiments 1 and 2), and the availability of referential gaze
(Experiment 2). In both experiments, children visually referenced the
experimenter more often while responding when the referent was
ambiguous. In Experiment 2, children also referenced more when there was
a novel referent and familiar distracter (and the referent could thus be
inferred), but only when referential gaze was unavailable. These results
suggest that children seek disambiguating social information on the
basis of graded uncertainty.}
\keywords{social referencing; help seeking; word learning; uncertainty.}

\begin{document}
\maketitle

Preschoolers quickly learn new concepts, rules, and language. They also
actively explore (Schulz \& Bonawitz, 2007) and ask questions
(Chouinard, Harris, \& Maratsos, 2007) in ways that seem targeted to
maximize learning. These behaviors appear to require children to monitor
epistemic states of ignorance and uncertainty, but young children have
generally been credited with limited ability to monitor mental states
(Sodian, Thoermer, Kristen, \& Perst, 2012). This presents a puzzle --
how do children acquire disambiguating information if they are not able
to monitor uncertainty? To better understand how young children actively
construct meaning from their environment, more evidence is needed about
the relation between epistemic uncertainty and information seeking in
early childhood.

Early research on metacognitive development suggested that young
children have limited capacities to reflect on their own thoughts or
knowledge. For example, 5-year-olds were found to have limited awareness
of their own ongoing thoughts (J. H. Flavell, Green, Flavell, Harris, \&
Astington, 1995), and preschoolers were unable to recognize that they
did not know all of the steps to complete a task unless prompted to act
it out (Markman, 1977). Later studies focused on young children's
ability to predict their performance in various tasks, often memory
tasks, and found that children tended to be overly optimistic about
their future performance (Schneider, 1998). However, these and other
studies tended to rely on children's explicit verbal reports about their
mental operations, often in a free-response format, which might
underestimate children's ability to track epistemic states.

More recently, researchers have investigated children's uncertainty
monitoring using tasks that do not require a verbal response from
children. In one study, 3- to 5-year-olds completed perceptual
discrimination and lexical identification tasks and reported on their
certainty in their choices using a pictorial confidence scale (Lyons \&
Ghetti, 2011). Preschoolers reported being less confident when they
responded incorrectly, suggesting that they were aware of their
likelihood of accuracy based on their epistemic states. In another
study, 3.5-year-olds completed a memory task in which they could opt out
of responding to individual trials (Balcomb \& Gerken, 2008). Children
performed worse on trials they had opted out of when they were forced to
answer them later on, suggesting that children used the opt-out option
strategically to avoid answering when they were uncertain about their
responses.

On the other hand, in a study with a similar structure that investigated
children's confidence in their memories for pictures, 3-year-olds were
equally confident for correct and incorrect answers, while 4- and
5-year-olds were more confident for their correct answers (Hembacher \&
Ghetti, 2014). Furthermore, only 5-year-olds were most confident about
their memories for images they had studied for longer. These results
suggest that there are limits to 3-year-olds' abilities to monitor
uncertainty, and that some cognitive states may be more easily reflected
on by young children than others. The developmental trend also suggests
that there is improvement in children's uncertainty monitoring between
ages 3 and 5.

In sum, preschool-aged children demonstrate an emerging but limited
awareness of epistemic uncertainty in laboratory tasks that do not
require a verbal or open-ended response. However, even the competencies
reported in these studies may underestimate children's ability to track
epistemic uncertainty and act on it. One possibility is that children's
tracking of uncertainty is apparent in their spontaneous
information-seeking behaviors before they can report on it explicitly.

In the following sections, we first outline an emerging body of research
that suggests that children actively gather data to support learning
from a young age, in a way that reflects sensitivity to epistemic
uncertainty. We then summarize previous work on children's ability to
selectively gather social information from credible sources and under
conditions of epistemic uncertainty.

\subsection{Active Learning in Early
Childhood}\label{active-learning-in-early-childhood}

Active learning refers broadly to learning behaviors and materials that
are initiated or generated by the learner themself, rather than by a
teacher (Gureckis \& Markant, 2012). Evidence for active learning during
early childhood comes from studies of overt question-asking and
help-seeking in preschoolers and toddlers, and behaviors that reflect
active information-seeking in infants. We review this evidence next.

\subsubsection{Spontaneous exploration in preschoolers and
toddlers}\label{spontaneous-exploration-in-preschoolers-and-toddlers}

First, there are studies that take inspiration from comparative
metacognition research and ask whether children's spontaneous
information-seeking behaviors track uncertainty. For example, Call and
Carpenter (2001) had 2-year-olds choose between several tubes to find a
hidden sticker. They found that the toddlers were more likely to peek
inside a tube before choosing when they had not seen the baiting of the
tubes compared to when they had, suggesting they were aware of their
knowledge or ignorance and selectively sought confirmatory evidence
before committing to a response when they did not know a sticker's
location. In another study, Goupil, Romand-Monnier, and Kouider (2016)
trained 20-month-olds to look at their parents if they needed help with
a memory task in which they had to locate a hidden toy. Toddlers were
more likely to seek help when they were completely ignorant of the toy's
location, and when the task was more difficult because the delay between
hiding and test was longer.

There is also evidence that young children selectively explore when
evidence is confounded (i.e., when existing evidence supports multiple
causal hypotheses). For example, Schulz \& Bonawitz (2007) found that
preschoolers spent more time playing with a toy when they witnessed
confounded evidence about its causal structure. Children who experienced
a demonstration of a toy in which two levers were simultaneously
depressed and caused two simultaneous events (a toy duck and puppet
popping up) spent more time playing with that toy compared to children
who saw a demonstration in which one lever was depressed at a time,
revealing the unique function of each lever. This finding is consistent
with the prediction that preschool-aged children track causal ambiguity
and spontaneously explore more when they do not have sufficient evidence
to isolate a cause for an effect.

Young children are also capable of independently generating the evidence
they need to distinguish between causal hypotheses. Sim and Xu (2017)
found that 2- and 3-year-olds were able to spontaneously discover first-
and second-order rules to activate a toy (e.g., the triangle block makes
the triangle toy work and toys are activated by shape-matched blocks)
during free-play. Preschoolers also identify the most informative
questions to confirm or disconfirm hypotheses (Ruggeri, Sim, \& Xu,
2017). These behaviors would be unlikely to occur if children did not
have some sensitivity to their epistemic states; they must track
uncertainty in order to intervene effectively and gather disambiguating
data.

\subsubsection{Information-seeking and uncertainty-tracking in
infants}\label{information-seeking-and-uncertainty-tracking-in-infants}

Infants have also been shown to selectively seek information under
conditions of ambiguity, and explore or allocate attention to
information sources that are most likely disambiguate. For example, in
one pair of studies, 7- and 8-month-olds attended longer to visual
displays (Kidd, Piantadosi, \& Aslin, 2012) or auditory stimuli (Kidd,
Piantadosi, \& Aslin, 2014) that were neither too complex nor too
simple, and thus afforded the greatest learning opportunity.
Twelve-month-old infants have also been shown to track the likelihood of
different events (e.g., a red ball exiting a container full of moving
red and blue balls) based on reasoning about multiple cues -- numerosity
(the number of red compared to blue balls), physical distance (the
distance from a red ball to the opening of the container) and time (how
long the scene was occluded before the ball left the container).
Intriguingly, infants' surprisal (looking time) tracked with uncertainty
about the likelihood of events -- for example, at intermediate occlusion
times, infants' looking time was graded with respect to distance and
numerosity in an additive fashion (Teglas et al., 2011).

While most studies of information-seeking in infancy have relied on
looking behavior, Stahl and Feigenson (2015) explored infants'
exploratory actions with toys that had either behaved normally or
appeared to violate physics (e.g., by hovering in midair). They found
that infants were more likely to explore those objects that had violated
physical expectations, and they explored in ways that were suited to
testing the physical property they had seen violated (e.g., picking up
and dropping the car that had hovered in midair; banging a toy that had
appeared to move through a solid wall). These results suggest that even
young infants selectively explore objects to disambiguate causal
properties that have been rendered uncertain by a surprising event.

In sum, there is a growing body of evidence that infants and young
children are active learners who explore based on their epistemic
states. These studies also suggest that spontaneous information-seeking
behaviors may be a fruitful behavioral index of children's sensitivity
to uncertainty in different learning contexts. However, much of this
research is confined to situations in which children must discover the
causal properties of interesting objects. Does epistemic uncertainty
drive other types of information seeking across learning domains and
information sources?

In particular, what is the role of uncertainty monitoring in children's
social learning? Infants and children learn a great deal from social
partners, through observation and direct pedagogy. But not all social
information is created equal, and learners must determine \emph{who} to
seek social information from, \emph{when} to seek it, and \emph{how} to
seek it. In the next section we outline previous research on the
selectivity of early social learning.

\subsection{Selective social learning}\label{selective-social-learning}

Although children appear to be biased towards trusting others' testimony
overall (Jaswal, Croft, Setia, \& Cole, 2010), there is evidence that
they are selective in their social learning. Children track the quality
of informants' testimony, and prefer to learn from people who have a
history of accuracy (Corriveau \& Harris, 2009; Sobel \& Kushnir, 2013).
For example, older preschoolers choose to learn the words for novel
objects from someone who has previously labeled objects correctly
compared to incorrectly (Koenig, Clement, \& Harris, 2009), and from
people who appear confident rather than uncertain (Sabbagh \& Baldwin,
2001). They also prefer to learn from people who have proven to be
honest rather than dishonest in the past (Li, Heyman, Xu, \& Lee,
2014){]}. Altogether, young children appear to be selective learners
with regard to the \emph{sources} of social information they seek.

Are children also selective about \emph{when} they seek social
information? That is, do they seek social information on the basis of
their uncertainty about their own mental representations? There is some
evidence to suggest that they do, at least in the context of solving
problems of varying difficulty. For example, Vredenburgh \& Kushnir
(2015) found that preschool-aged children selectively sought help with a
task when they were either less competent at the task or when they
reached a difficult step. Coughlin, Hembacher, Lyons and Ghetti (2014)
found that preschoolers were more likely to seek help on trials of a
perceptual discrimination task for which they reported lower confidence
in a separate session in which help was not available. Additionally, an
analysis of a corpus of parent-child conversations showed that
preschoolers tend to ask questions that are directed to resolve gaps in
knowledge, and they ask follow-up questions if they receive
uninformative responses (Chouinard et al., 2007).

Overt help-seeking is only one form of social information-gathering.
Infants and children also engage in \emph{social referencing}, or
checking a social partner's face for gaze direction and/or reactions to
stimuli or events (Walden \& Ogan, 1988). There is plentiful evidence
that infants reference social partners to determine the safety of
objects or actions; for example, infants reference their parents' faces
before crossing a bridge or drop-off of uncertain slope or depth (Sorce,
Emde, Campos, \& Klinnert, 1985; Tamis LeMonda et al., 2008). Infants
also check adults' faces when exposed to potentially dangerous stimuli,
such as barking sounds or spiders (Striano \& Rochat, 2000; Zarbatany \&
Lamb, 1985). In summary, infants and young children reference social
information in response to uncertainty about the physical affordances or
safety of a situation.

Social referencing is especially critical for language acquisition. By
the second year of life infants follow a speaker's gaze and map labels
to objects on the basis of gaze direction (Baldwin, 1991). There is also
evidence that infants' propensity for gaze-following predicts later
language development, highlighting the importance of this behavior for
learning. Furthermore, word learning is more successful under joint
attention, which involves mutual knowledge about a shared focus of
attention (Baldwin, 1993; Tomasello \& Farrar, 1986).\footnote{We use
  the term ``social referencing'' to indicate the generic behavior of
  seeking social information whether it is specifically referential (as
  in gaze following) or otherwise.}

However, there is still only limited evidence about whether infants and
young children are \emph{selective} in seeking out these social cues to
reference. Do children orient towards a speaker indiscriminately, or do
they selectively look for social cues when reference cannot be
determined without them? It could be that social referencing is
typically not costly enough to require selectivity. People shift gaze
several times per second, and social cues are intrinsically salient
(Csibra \& Gergely, 2006). Thus, children may sample social information
frequently regardless of its potential to disambiguate reference.

Another possibility is that uncertainty signals related to knowledge
representations are too weak to drive information-seeking behaviors in
young children. Consistent with this possibility, as discussed
previously, studies of early metacognition have typically found that
young children are overconfident in their knowledge and performance
predictions, even when they are calibrated overall to accuracy (Coughlin
et al., 2014; Hembacher \& Ghetti, 2014; Schneider, 1998). If children
are not aware of the need for disambiguating information, they may not
selectively reference social information on ambiguous trials.

At least one study has previously investigated the selectivity of
infants' social referencing during word learning. Infants heard an
experimenter produce a label for an object in the presence of one or two
novel objects (Vaish, Demir, \& Baldwin, 2011). Infants looked up at the
experimenter more often when there were two objects present, and the
referent was thus ambiguous. While this study provides encouraging
evidence that infants' social referencing may be sensitive to
referential uncertainty, there are questions that remain unanswered.
Since referential ambiguity was manipulated through the number of items
present, it is unclear whether infants' additional looking was based on
epistemic uncertainty or the complexity of the environment (i.e., the
presence of multiple objects). In the current study, we seek to
replicate these findings with preschool-aged children while controlling
for the number of objects present.

\subsection{Current study}\label{current-study}

The present study builds on previous work suggesting that preschoolers
are active and selective in their social learning. Here, we ask whether
preschoolers reference a speaker more frequently when the referent of
their speech is ambiguous. To this end, we adapt the procedure used by
Vaish et al. (2011) for use with young children, controling for the
number of objects present. In Experiment 1 we include trials with two
novel objects (referentially ambiguous), two familiar objects
(referentially unambiguous), one novel object (referentially
unambiguous) and one familiar object (referentially unambiguous). In
Experiment 2, we additionally manipulate the amount of referential
evidence available by manipulating whether or not the speaker's gaze is
informative, allowing us to investigate whether preschoolers' social
referencing is sensitive to graded uncertainty.

There are several reasons to investigate children's sensitivity to
graded evidence. First, probabilistic models of cognition and learning
require that individuals be able to track the amount of evidence for
alternative beliefs (and corresponding uncertainty), perhaps through
heuristics such as ``win-stay lose-sample'' (E. Bonawitz, Denison,
Gopnik, \& Griffiths, 2014). If this is true of preschool-aged children,
their information-gathering behaviors should be graded with respect to
the amount of evidence for a belief. Critically, individuals may not
adjust their information seeking in a graded fashion with respect to
uncertainty, but situations characterized by intermediate referential
uncertainty (e.g., because only one cue is present) should elicit
intermediate amounts of information-seeking in the aggregate (Daw \&
Courville, 2007).

More generally, a complete theory of early metacognition and learning
should encompass children's ability to respond appropriately to graded
evidence. Metacognitive theories assume that effective behavioral
regulation and response selection rely on individuals' abilities to
track the likelihood of being correct based not only on complete
knowledge or ignorance, but on graded evidence (Lyons \& Zelazo, 2011).
For example, having more information in support of a belief should make
individuals more willing to share that information with others, and
having less information should motivate help-seeking (Ghetti, Hembacher,
\& Coughlin, 2013).

In the current study, we examine social referencing across the time
course of an event that includes a speaker labeling an object and
requesting that the child put the object in a bucket. Examining
children's social referencing at different points in this interaction
allows us to determine whether children differentially reference a
speaker depending on the social information available at different
moments. For example, children might reference a speaker during labeling
to discover the gaze direction of the speaker, or during their choice of
an object to glean the speaker's evaluation of their choice.

\begin{CodeChunk}
\begin{figure*}[b]

{\centering \includegraphics[width=5in,height=3in]{figs/design-1} 

}

\caption[Study design for experiments 1 and 2]{Study design for experiments 1 and 2.}\label{fig:design}
\end{figure*}
\end{CodeChunk}

\section{Experiment 1}\label{experiment-1}

In Experiment 1, we examined whether children would visually reference a
speaker more often when the speaker produced a referentially ambiguous
label compared to an unambiguous label. Children sat across from an
experimenter who labeled an object on the table between them (Figure
\ref{fig:design}). The experimenter then asked the child to place the
named object in a bucket. Across trials, there were either one or two
objects on the table, which were either familiar or novel to the child.
This design allowed us to test whether merely having more than one
object present is sufficient to increase social referencing (which could
not be ruled out by Vaish et al.), or if referential ambiguity (and thus
epistemic uncertainty) is the underlying factor. If the latter is true,
we expected children to increase their looking to the experimenter only
on trials with two unfamiliar objects, when the object-label mapping was
not known and could not be inferred.

We were interested in the amount of social referencing children
exhibited across the trial. We considered four different phases of each
trial based on the notion that children might expect different social
information at different stages of the task. Specifically, we predicted
that children might expect the speaker's gaze direction to be
informative during the labeling itself, as speakers tend to look at
objects they refer to. We predicted that later in the trial, as children
reached for an object and placed it in the bucket, they might expect
evaluative feedback about their choice (e.g., facial expressions of
encouragement or discouragement).

\subsection{Methods}\label{methods}

\subsubsection{Participants}\label{participants}

We recruited a planned sample of 80 children ages 2-5 years from the
Children's Discovery Museum in San Jose, California.\footnote{Planned
  sample size, exclusion criteria, and analysis plan preregistered at
  \url{https://osf.io/y7mvt}.} The sample included 20 2-year-olds (mean
age 32.0 months), 20 3-year-olds (mean age 42.7 months), 20 4-year-olds
(mean age 55.8 months), and 20 5-year-olds (mean age 65.2 months). An
additional 20 children participated but were removed from analyses
because they heard English less than 75\% of the time at home (\emph{n}
= 10), because they were unable to complete at least half of the trials
in the task (\emph{n} = 4), because of parental interference (\emph{n} =
1), or due to experimenter or technical errors (\emph{n} = 5).

\subsubsection{Stimuli and Design}\label{stimuli-and-design}

Children were presented with one or two objects, heard a label, and were
asked to put the labeled object in a bucket. Half of the objects were
selected to be familiar to children (e.g., a cow) and half were selected
to be novel (e.g., a nozzle). There were four possible trial types based
on the number and familiarity of the objects present: one familiar
object (F), one novel object (N), two familiar objects (FF), and two
novel objects (NN). There were three trials of each type, for a total of
twelve trials. Trial types were presented sequentially in an order that
was counterbalanced across participants. The assignment of individual
objects to trial types was counterbalanced. On F/FF trials, the familiar
label for the target object was used (e.g., ``cow''). On N/NN trials, a
novel label was used (e.g., ``dawnoo'').

The critical manipulation was of referential ambiguity; F and FF trials
were referentially unambiguous, as children were expected to be certain
about the objects and their labels. Similarly, on N trials, children
were expected to be certain about the label referent as there was only
one option. However, NN trials were referentially ambiguous, as the
novel label could apply to either novel object.

Throughout the task, the experimenter never gazed at the object they
were labeling, or responded to children's verbal or non-verbal bids for
help by indicating the correct object. Thus, children were expected to
remain uncertain about the referent throughout the trial when two novel
objects were present.

\subsubsection{Procedure}\label{procedure}

Throughout the study, the child sat at one end of a large circular
table, and the experimenter stood at the opposite end. Each trial of the
task proceeded as follows: the experimenter placed one or two objects on
the left and/or right sides of the table, out of reach of the child so
that the child could not interact with the toys during the labeling
event. For one-object trials, the location of the object (left or right)
alternated between trials.

After placing the objects, the experimenter said ``Hey look, there's a
(target) here.'' The experimenter gazed at the center of the table
rather than the object they labeled. The experimenter waited
approximately two seconds based on a visual metronome placed within view
before saying, ``Can you put the (target) in the bucket?'' They then
pushed the object(s) forward within reach of the child, and placed a
plastic bucket in the center of the table, also within reach of the
child. Prior to the twelve experimental trials, there were two training
trials: a F trial and a FF trial, to acquaint the child with the
procedure. A camera placed to the side of the experimenter captured the
participant's face, so that looking behavior could be coded from video.

\subsubsection{Coding procedure and analytic
plan}\label{coding-procedure-and-analytic-plan}

Videos were coded using DataVyu software (\url{http://datavyu.org}). For
each participant, we coded the \emph{number of times} they referenced
the experimenter throughout each trial. An alternative analytic option
would be to simply code \emph{whether or not} children looked at the
experimenter. However, during piloting, we found that most children
looked up to the experimenter at least once while they were labeling the
object, suggesting that a binary measure of looking would not be
meaningful.

Because we were interested in the precise circumstances in which
children feel uncertain enough to reference a speaker, we coded the
number of looks that occurred during four distinct phases of the trial:
a \emph{label} phase, which began at the utterance of the label and
ended when the experimenter began to slide the objects, a \emph{slide}
phase, in which the experimenter slid the object(s) into the child's
reach, a \emph{planning} phase, which began at the end of the slide and
ended when the child touched an object, and a \emph{response} phase,
which began when the child touched an object and ended when the child
released the object into the bucket.

We also noted any trials that should be excluded from analyses due to
the child's interference with the timing of the trial (e.g., talking
over the experimenter), experimenter error, or outside distractions that
interfered with the timing of the trial (e.g., noise from a sibling).
These trial-wise exclusion criteria were preregistered. In total, we
excluded 1.4\% of trials from 2-year-olds, 2.0\% of trials from
3-year-olds, 1.9\% of trials from 4-year-olds, and 2.0\% of trials from
5-year-olds on this basis.

A second coder independently scored the number of looks for one third of
the trials for each participant to establish reliability. Inter-rater
reliability for the number of looks in each phase was high, intraclass
correlation \emph{r} = .97, \emph{p}\textless{}.001.

Table \ref{tab:phases} displays the average durations of each of the
four phases. The \emph{label} and \emph{response} phases are longer on
average than the \emph{planning} and \emph{slide} phases. However, note
that we are interested in comparing the amount of social referencing
across ambiguity conditions and not across phases directly.

To quantify the effects of the number and familiarity of objects on
children's looking, along with any developmental trends, we planned to
fit a linear mixed-effects regression, beginning with the following
structure and trimming according to standard laboratory
procedures\footnote{Standard laboratory analytic procedures available at
  \url{https://osf.io/zqzsu/wiki/home/}}:
\texttt{number\ of\ looks\ \textasciitilde{}\ number\ of\ objects\ *\ familiarity\ *\ phase\ *\ age\ in\ months\ +\ (number\ of\ objects\ *\ familiarity\ \textbar{}\ subject)}.
This model specification was preregistered, as noted above.

\subsection{Results and Discussion}\label{results-and-discussion}

\subsubsection{Accuracy}\label{accuracy}

\begin{CodeChunk}
\begin{figure*}[b]

{\centering \includegraphics{figs/acc_e1-1} 

}

\caption[Accuracy for FF trials in Experiment 1]{Accuracy for FF trials in Experiment 1. Error bars are 95 percent confidence intervals.}\label{fig:acc_e1}
\end{figure*}
\end{CodeChunk}

\begin{table}[b]
\centering
\begin{tabular}{rllrr}
  \hline
 & Experiment & Phase & Mean Duration & SD \\ 
  \hline
1 & 1 & label & 5072.84 & 899.16 \\ 
  2 & 1 & slide & 872.43 & 196.83 \\ 
  3 & 1 & planning & 1253.85 & 2215.56 \\ 
  4 & 1 & response & 3646.40 & 4510.01 \\ 
   \hline
5 & 2 & label & 5239.23 & 748.05 \\ 
  6 & 2 & slide & 818.95 & 1307.27 \\ 
  7 & 2 & planning & 790.88 & 10327.45 \\ 
  8 & 2 & response & 4084.56 & 5032.37 \\ 
   \hline
\end{tabular}
\caption{Phase Durations} 
\label{tab:phases}
\end{table}

We examined children's accuracy for those trials in which a correct
response was possible (i.e., FF trials). Children sometimes put two
items in the bucket (2-year-olds: 62.5\% of trials; 3-year-olds: 27.9\%;
4-year-olds: 15.0\%; 5-year-olds: 17.5\%), despite instructions to
choose only one. If they put the correct item in the bucket first
followed by the incorrect object, we counted their response as correct.
If they put the incorrect object in first, we counted their response as
incorrect. Children also occasionally declined to choose an item (0.8\%
of trials); these trials are excluded from accuracy analyses. Children's
accuracy is displayed in Figure \ref{fig:acc_e1}. Overall children
generally chose the correct item for FF trials, indicating that they
understood the task and were motivated to answer correctly. While 3- to
5-year-olds performed close to ceiling (92\% - 99\%), 2-year-olds were
less accurate (76\%), but still performed significantly above chance
(\emph{t}(19) = 23.0 , \emph{p} \textless{}.001).

\begin{CodeChunk}
\begin{figure*}[b]

{\centering \includegraphics[width=5.75in,height=4.35in]{figs/results_e1-1} 

}

\caption[Results of Experiment 1]{Results of Experiment 1. Number of looks to the experimenter based on phase, the number and familiarity of objects present, and age. Age in months was entered as a continuous variable in regression models but is presented here as a categorical variable for visual simplicity. Error bars are 95 percent confidence intervals.}\label{fig:results_e1}
\end{figure*}
\end{CodeChunk}

\begin{table}[b]
\centering
\begin{tabular}{lrrrrl}
 Predictor & Estimate & Std. Error & $t$ value & $p$ value &  \\ 
  \hline
Intercept & 1.38 & 0.06 & 22.58 & $<$ .001 & *** \\ 
  Num objects (2) & -0.06 & 0.06 & -1.02 & 0.31 &  \\ 
  Familiarity (N) & 0.07 & 0.06 & 1.16 & 0.25 &  \\ 
  Age & 0.01 & 0.00 & 1.85 & 0.07 & . \\ 
  Num objects (2) * Familiarity (N) & -0.06 & 0.09 & -0.74 & 0.46 &  \\ 
  Num objects (2) * Age & -0.01 & 0.00 & -1.18 & 0.24 &  \\ 
  Familiarity (N) * Age & -0.01 & 0.00 & -1.17 & 0.24 &  \\ 
  Num objects (2) * Familiarity (N) * Age & -0.00 & 0.01 & -0.57 & 0.57 &  \\ 
   \hline
\end{tabular}
\caption{Predictor estimates with standard errors and significance information for a linear mixed-effects model predicting social referencing in the label phase in Experiment 1.} 
\label{tab:exp1_l_reg}
\end{table}

\begin{table}[b]
\centering
\begin{tabular}{lrrrrl}
 Predictor & Estimate & Std. Error & $t$ value & $p$ value &  \\ 
  \hline
Intercept & 0.12 & 0.02 & 5.06 & $<$ .001 & *** \\ 
  Num objects (2) & 0.02 & 0.03 & 0.55 & 0.58 &  \\ 
  Familiarity (N) & 0.04 & 0.03 & 1.38 & 0.17 &  \\ 
  Age & -0.00 & 0.00 & -0.44 & 0.66 &  \\ 
  Num objects (2) * Familiarity (N) & -0.07 & 0.04 & -1.69 & 0.09 & . \\ 
  Num objects (2) * Age & 0.00 & 0.00 & 0.20 & 0.84 &  \\ 
  Familiarity (N) * Age & -0.00 & 0.00 & -0.18 & 0.86 &  \\ 
  Num objects (2) * Familiarity (N) * Age & -0.00 & 0.00 & -0.84 & 0.4 &  \\ 
   \hline
\end{tabular}
\caption{Predictor estimates with standard errors and significance information for a linear mixed-effects model predicting social referencing in the slide phase in Experiment 1.} 
\label{tab:exp1_s_reg}
\end{table}

\begin{table}[b]
\centering
\begin{tabular}{lrrrrl}
 Predictor & Estimate & Std. Error & $t$ value & $p$ value &  \\ 
  \hline
Intercept & 0.03 & 0.02 & 1.48 & 0.14 &  \\ 
  Num objects (2) & 0.01 & 0.04 & 0.26 & 0.8 &  \\ 
  Familiarity (N) & 0.00 & 0.03 & 0.12 & 0.9 &  \\ 
  Age & -0.00 & 0.00 & -0.92 & 0.36 &  \\ 
  Num objects (2) * Familiarity (N) & 0.20 & 0.05 & 4.34 & $<$ .001 & *** \\ 
  Num objects (2) * Age & 0.00 & 0.00 & 0.54 & 0.59 &  \\ 
  Familiarity (N) * Age & 0.00 & 0.00 & 0.73 & 0.47 &  \\ 
  Num objects (2) * Familiarity (N) * Age & 0.00 & 0.00 & 1.42 & 0.16 &  \\ 
   \hline
\end{tabular}
\caption{Predictor estimates with standard errors and significance information for a linear mixed-effects model predicting social referencing in the planning phase in Experiment 1.} 
\label{tab:exp1_p_reg}
\end{table}

\begin{table}[b]
\centering
\begin{tabular}{lrrrrl}
 Predictor & Estimate & Std. Error & $t$ value & $p$ value &  \\ 
  \hline
Intercept & 0.22 & 0.04 & 5.35 & $<$ .001 & *** \\ 
  Num objects (2) & 0.05 & 0.06 & 0.96 & 0.34 &  \\ 
  Familiarity (N) & 0.01 & 0.05 & 0.18 & 0.86 &  \\ 
  Age & -0.00 & 0.00 & -0.20 & 0.84 &  \\ 
  Num objects (2) * Familiarity (N) & 0.60 & 0.07 & 8.21 & $<$ .001 & *** \\ 
  Num objects (2) * Age & 0.00 & 0.00 & 0.85 & 0.4 &  \\ 
  Familiarity (N) * Age & 0.00 & 0.00 & 0.55 & 0.58 &  \\ 
  Num objects (2) * Familiarity (N) * Age & 0.01 & 0.01 & 1.34 & 0.18 &  \\ 
   \hline
\end{tabular}
\caption{Predictor estimates with standard errors and significance information for a linear mixed-effects model predicting social referencing in the response phase in Experiment 1.} 
\label{tab:exp1_r_reg}
\end{table}

\subsubsection{Social referencing}\label{social-referencing}

Results of Experiment 1 are presented in Figure \ref{fig:results_e1}. To
test our prediction that referential ambiguity (i.e., having two novel
objects) would produce more social referencing, we fit mixed-effects
linear regression models separately for each phase with the following
structure:
\texttt{number\ of\ looks\ \textasciitilde{}\ number\ of\ objects\ *\ familiarity\ *\ age\ in\ months\ +\ (number\ of\ objects\ +\ familiarity\ \textbar{}\ subject)}.
A single model with phase as a factor did not converge, and the model
was subsequently trimmed according to our standard laboratory analytic
procedures.

We did not find any main or interaction effects of number of objects,
familiarity, or age on number of looks during the \emph{label} or
\emph{slide} phases. Thus, mere novelty or the presence of multiple
objects was not enough to increase social referencing. However, we found
an interaction effect of number of objects and familiarity during the
\emph{planning} (\(\beta\) = 0.2, \emph{p} \textless{} .001) and
\emph{response} phases (\(\beta\) = 0.6, \emph{p} \textless{} .001),
such that NN trials were associated with more looking.

There was no interaction with age in any phase.\footnote{\url{https://github.com/emilyfae/socref_uncert}}
Since an age trend in the \emph{planning} phase was apparent in the plot
(Figure \ref{fig:results_e1}), we conducted a post-hoc mixed-effects
linear regression with age group (2- and 3-year-olds vs.~4- and
5-year-olds) predicting the number of looks in the \emph{planning}
phase:
\texttt{number\ of\ looks\ \textasciitilde{}\ number\ of\ objects\ *\ familiarity\ *\ age\ group\ +\ (number\ of\ objects\ +\ familiarity\ \textbar{}\ subject)}.
We found a significant interaction of number of objects and familiarity
(\(\beta\) = 0.3, \emph{p} \textless{} .001), and a significant 3-way
interaction between number of objects, familiarity, and age group, such
that younger children looked at the speaker less than older children for
NN trials. Although this was a post-hoc analysis, it suggests that there
may be a trend for children to respond more quickly to uncertainty as
they grow older. This could mean that they detect uncertainty more
quickly, or that they are more proactive in seeking disambiguating
information prior to initiating a response.

\begin{CodeChunk}
\begin{figure*}[b]

{\centering \includegraphics{figs/hist_e1-1} 

}

\caption[Distribution of the number of looks to the speaker in each phase]{Distribution of the number of looks to the speaker in each phase.}\label{fig:hist_e1}
\end{figure*}
\end{CodeChunk}

As discussed previously, an alternative analytic approach would be to
fit a logistic mixed-effects regression predicting \emph{whether or not}
children look to the speaker in each phase, rather than the number of
times they do so. In support of this approach, the distribution of looks
is not normal for the slide, planning, and response phases, with the
majority of trials containing no looks to the speaker (Figure
\ref{fig:hist_e1}). To address this issue, we additionally fit separate
logistic mixed-effects regressions with the following structure:
\texttt{look\ (yes\ or\ no)\ \textasciitilde{}\ number\ of\ objects\ *\ familiarity\ *\ age\ group\ +\ (number\ of\ objects\ +\ familiarity\ \textbar{}\ subject)}.
A single model with phase as a factor did not converge.

The results of these analyses were similar to those of the linear
models; number of objects and familiarity interacted in the
\emph{planning} and \emph{response} phases such that there was more
likely to be a look when there were two novel objects, (\(\beta\)s =
1.4, 2.0, \emph{p}s \textless{}.05 \& \textless{}.001). In addition,
there was a significant interaction of number and familiarity of objects
in the \emph{slide} phase, such that NN trials were less likely to be
associated with a look (\(\beta\) = -0.9, \emph{p} \textless{}.05). This
latter finding was not predicted, but could reflect children's tendency
to look more at novel stimuli, which would trade off with looking at the
speaker, particularly when there is no utility to looking at the
speaker, as when the objects are being slid across the table.

In summary, children looked to the speaker more often when planning and
executing a response under uncertainty. These results suggest that
children were aware that they did not have sufficient knowledge to
answer independently, and referenced the speaker to resolve this
uncertainty.

Notably, and in contrast to Vaish et al., we did not find the expected
effect of referential ambiguity in the \emph{label} phase. It is
possible that children failed to predict that they would need more
information until later in the trial, when they were actually faced with
making a decision. Another possibility is that children's looking was at
ceiling during the labeling phase, perhaps because children tend to look
at someone who is speaking regardless of the need for referential
disambiguation.

A third possibility is that this finding is an artifact of our design,
in which the speaker gazed at the center of the table rather than the
referent of the label. Children may have realized that the speaker's
gaze direction during labeling was not informative. Alternatively,
children may have found it strange to interact with a speaker who did
not gaze at the object they were labeling, which may have produced
unnatural patterns of social referencing. Experiment 2 tests these
possibilities and examines whether children's social referencing is
sensitive to graded uncertainty.

\section{Experiment 2}\label{experiment-2}

Experiment 2 was designed to achieve several goals; first, we aimed to
replicate the finding from Experiment 1 that children reference a social
partner on the basis of referential ambiguity while executing a
decision. Second, we tested whether children's social referencing is
graded with respect to graded evidence about a label's referent.

To this end, we manipulated two sources of referential evidence. First,
we added trials with 1 novel and 1 familiar object (FN) and a novel
label. This condition contains evidence about reference since the
familiar item can be excluded (Markman \& Wachtel, 1988), but may not be
as conclusive as trials with a familiar target. Thus, we predicted that,
in the aggregate, children would show the most looking to the speaker in
the NN condition, the least in the FF condition, and an intermediate
amount in the FN condition.

Second, we manipulated between participants whether or not the speaker
gazed at the objects they referred to, and thus, whether or not their
gaze was an informative cue to reference. We predicted that having
access to referential gaze as an informative cue would make children
less likely to reference the speaker during their decision, but perhaps
more likely to reference the speaker during labeling. Critically, this
also allowed us to test whether children's social referencing is
selective on the basis of referential ambiguity during labeling if the
speaker's gaze is informative, addressing an interpretive issue in
Experiment 1.

Since we did not observe any difference between F and N trials in
Experiment 1, we eliminated single-object trials, leaving the FF and NN
trials. Additionally, we restricted the sample to 3- and 4-year-olds, as
there was no effect of age in our preregistered analyses in Experiment
1, but a trend for younger children to engage in selective referencing
earlier in the trial than older children.

\subsection{Methods}\label{methods-1}

\subsubsection{Participants}\label{participants-1}

We recruited a planned sample of 80 children ages 3-4 years from the
Children's Discovery Museum in San Jose, California.\footnote{Planned
  sample size, exclusion criteria, and analysis plan (including model
  specification) preregistered at \url{https://osf.io/y7mvt/}.} The
sample included 40 3-year-olds (mean age 42.9 months) and 40 4-year-olds
(mean age 53.5 months). An additional 20 children participated but were
removed from analyses because they heard English less than 75\% of the
time at home (\emph{n} = 9), because they were unable to complete at
least half of the trials in the task (\emph{n} = 7), or due to
experimenter or technical errors (\emph{n} = 4).

\subsubsection{Stimuli and Design}\label{stimuli-and-design-1}

The stimuli and design were similar to Experiment 1 but included three
trial types: FF, NN, and FN. There were four of each trial type,
totaling twelve trials. In addition, we manipulated the experimenter's
gaze behavior between participants. For half of the participants, the
experimenter gazed at the center of the table while labeling objects
(uninformative gaze); for the remaining half, they gazed directly at the
objects they labeled (informative gaze).

\subsubsection{Procedure}\label{procedure-1}

The experimental and coding procedures were identical to Experiment 1,
except that there were three practice trials (two familiar trials and
one novel trial). We chose this approach so that children could
experience both familiar and novel stimuli during practice, but would
not be discouraged by an overly difficult practice session.

We again noted trials that should be excluded based on the same criteria
as in Experiment 1. We excluded 1.9\% of trials from 3-year-olds and no
trials from 4-year-olds. Inter-rater reliability for the number of looks
in each phase was again high, intraclass correlation \emph{r} = .97,
\emph{p}\textless{}.001.

The mean durations of the phases for Experiment 2 are presented in Table
\ref{tab:phases}. They varied in length according to the same pattern as
in Experiment 1.

\subsection{Results and Discussion}\label{results-and-discussion-1}

\subsubsection{Accuracy}\label{accuracy-1}

\begin{CodeChunk}
\begin{figure*}[b]

{\centering \includegraphics{figs/acc_e2-1} 

}

\caption[Accuracy for trials with a correct answer available (FF and FN, and all gaze trials) in Experiment 2]{Accuracy for trials with a correct answer available (FF and FN, and all gaze trials) in Experiment 2. Error bars are 95 percent confidence intervals.}\label{fig:acc_e2}
\end{figure*}
\end{CodeChunk}

\begin{table}[b]
\centering
\begin{tabular}{lrrrrl}
 Predictor & Estimate & Std. Error & $t$ value & $p$ value &  \\ 
  \hline
Intercept & 0.11 & 0.09 & 1.21 & 0.23 &  \\ 
  Acc(Y) & 0.01 & 0.09 & 0.06 & 0.95 &  \\ 
  Age & -0.00 & 0.01 & -0.30 & 0.76 &  \\ 
  Phase(Label) & 0.96 & 0.12 & 8.00 & $<$ .001 & *** \\ 
  Phase(Planning) & 0.10 & 0.12 & 0.81 & 0.42 &  \\ 
  Phase(Response) & 1.31 & 0.12 & 10.91 & $<$ .001 & *** \\ 
  Acc(Y) * Age & 0.00 & 0.01 & 0.21 & 0.84 &  \\ 
  Acc(Y) * Phase(Label) & 0.41 & 0.12 & 3.31 & $<$ .001 & *** \\ 
  Acc(Y) * Phase(Planning) & -0.17 & 0.12 & -1.33 & 0.18 &  \\ 
  Acc(Y) * Phase(Response) & -0.99 & 0.12 & -7.99 & $<$ .001 & *** \\ 
  Age * Phase(Label) & 0.04 & 0.02 & 1.90 & 0.06 & . \\ 
  Age * Phase(Planning) & 0.02 & 0.02 & 0.99 & 0.32 &  \\ 
  Age * Phase(Response) & 0.02 & 0.02 & 0.98 & 0.33 &  \\ 
  Acc(Y) * Age * Phase(Label) & -0.02 & 0.02 & -1.16 & 0.25 &  \\ 
  Acc(Y) * Age * Phase(Planning) & -0.02 & 0.02 & -1.00 & 0.32 &  \\ 
  Acc(Y) * Age * Phase(Response) & -0.03 & 0.02 & -1.25 & 0.21 &  \\ 
   \hline
\end{tabular}
\caption{Predictor estimates with standard errors and significance information for a linear mixed-effects model predicting social referencing based on accuracy in Experiment 2.} 
\label{tab:exp2acc_reg}
\end{table}

Children's accuracy for trials with a correct answer (i.e., FF, FN, and
all trials with gaze) was calculated using the same criteria as in
Experiment 1 (Figure \ref{fig:acc_e2}). As in Experiment 1, children
sometimes put two items in the bucket (3-year-olds: 0.0\% of trials;
4-year-olds: 0.0\%). If they put the correct object in first, the trial
was counted as accurate. To quantify the effects of familiarity, gaze,
and age on accuracy, we fit the following mixed-effects logistic
regression model:
\texttt{correct\ \textasciitilde{}\ trial\ type\ *\ gaze\ +\ age\ in\ months\ +\ (1\textbar{}\ subject)}.
Accuracy was significantly lower for novel (\(\beta\) = -2.3, \emph{p}
\textless{} .001) and mutual trials (\(\beta\) = -2.2, \emph{p}
\textless{} .001) compared to familiar trials, and trial type interacted
with gaze condition such that accuracy was significantly higher in the
gaze condition for novel (\(\beta\) = 3.4, \emph{p} \textless{} .001)
and mutual trials (\(\beta\) = 1.1, \emph{p} \textless{} .001). Age did
not significantly predict accuracy (\(\beta\) = 0.2, \emph{p} = .12).

\subsubsection{Social referencing and referential
ambiguity}\label{social-referencing-and-referential-ambiguity}

Children's social referencing based on trial type and gaze condition are
presented in Figure \ref{fig:results_e2}. To quantify the main and
interactive effects of familiarity, gaze informativity, phase, and age
on social referencing, we fit a mixed-effects linear regression model
with the following structure:
\texttt{number\ of\ looks\ \textasciitilde{}\ familiarity\ *\ age\ in\ months\ *\ gaze\ *\ phase\ +\ (familiarity\ \textbar{}\ subject)}.
In contrast to Experiment 1, a model with phase as a predictor
converged.

First, do children reference a speaker more often when the objects and
label are novel? Phase interacted with familiarity such that the
\emph{response} phase of NN trials was associated with significantly
more looks (\(\beta\) = 0.3, \emph{p} \textless{} .001). This result is
consistent with results from Experiment 1. However, in contrast to
Experiment 1, the number of looks was not significantly greater for NN
trials in the \emph{planning} phase.

We were also interested in whether FN trials would elicit an
intermediate amount of looking. We observed a three-way interaction of
familiarity, gaze, and phase, such that the response phase of FN trials
in the no-referential-gaze condition was associated with significantly
more looks (\(\beta\) = -0.2, \emph{p} \textless{} .01). Thus, FN trials
were associated with greater looking only when the experimenter did not
provide informative gaze. This finding is intriguing given that children
should be able to solve FN trials without gaze information. Instead,
these results suggest that children remain relatively uncertain while
making a decision if excluding the familiar object is their only cue to
reference, but this uncertainty is resolved if the speaker's gaze is
informative. On the other hand, informative gaze during labeling did not
lessen social referencing for novel trials, suggesting that gaze
information alone was not sufficient to reduce uncertainty. Instead,
children required both types of evidence to feel certain about their
response.

Finally, we observed a four-way interaction such that the
\emph{response} phase of novel trials in the gaze condition was
associated with more looking with increasing age (\(\beta\) = 0.0,
\emph{p} \textless{} .01), suggesting that preschoolers may become more
selective in their social referencing as they get older. It may be that
children improve in their ability to monitor the need for disambiguating
information, or they may become more likely to recognize that social
information can be a source of disambiguation.

We did not observe selective social referencing during the \emph{label}
phase, even when referential gaze was available. This result rules out
the possibility that children are less selective during labeling because
they learned that gaze direction was not informative. Instead, children
may attend to someone who is speaking regardless of the need for
disambiguation.

\begin{CodeChunk}
\begin{figure*}[b]

{\centering \includegraphics[width=5.75in,height=3.5in]{figs/results_e2-1} 

}

\caption[Results of Experiment 2]{Results of Experiment 2. Number of looks to the experimenter based on phase, trial type, gaze condition, and age. Age in months was entered as a continuous variable in regression models but is presented here as a categorical variable for visual simplicity. Error bars are 95 percent confidence intervals.}\label{fig:results_e2}
\end{figure*}
\end{CodeChunk}

\begin{table}[b]
\centering
\begin{tabular}{lrrrrl}
 Predictor & Estimate & Std. Error & $t$ value & $p$ value &  \\ 
  \hline
Intercept & 0.14 & 0.06 & 2.62 & 0.01 & ** \\ 
  Trial type(FN) & -0.09 & 0.07 & -1.27 & 0.21 &  \\ 
  Trial type(NN) & -0.02 & 0.08 & -0.29 & 0.77 &  \\ 
  Age & 0.00 & 0.01 & 0.27 & 0.79 &  \\ 
  Gaze(Y) & -0.01 & 0.08 & -0.09 & 0.93 &  \\ 
  Phase(Label) & 1.36 & 0.07 & 18.71 & $<$ .001 & *** \\ 
  Phase(Planning) & -0.11 & 0.07 & -1.54 & 0.12 &  \\ 
  Phase(Response) & 0.10 & 0.07 & 1.38 & 0.17 &  \\ 
  Trial type(FN) * Age & -0.01 & 0.01 & -0.80 & 0.43 &  \\ 
  Trial type(NN) * Age & -0.01 & 0.01 & -0.66 & 0.51 &  \\ 
  Trial type(FN) * Gaze & 0.08 & 0.10 & 0.77 & 0.44 &  \\ 
  Trial type(NN) * Gaze & -0.01 & 0.11 & -0.10 & 0.92 &  \\ 
  Age * Gaze & -0.00 & 0.01 & -0.35 & 0.72 &  \\ 
  Trial type(FN) * Phase(Label) & 0.02 & 0.10 & 0.24 & 0.81 &  \\ 
  Trial type(NN) * Phase(Label) & -0.02 & 0.10 & -0.23 & 0.82 &  \\ 
  Trial type(FN) * Phase(Planning) & 0.21 & 0.10 & 2.01 & 0.04 & * \\ 
  Trial type(NN) * Phase(Planning) & 0.28 & 0.10 & 2.77 & 0.01 & ** \\ 
  Trial type(FN) * Phase(Response) & 0.53 & 0.10 & 5.09 & $<$ .001 & *** \\ 
  Trial type(NN) * Phase(Response) & 0.68 & 0.10 & 6.52 & $<$ .001 & *** \\ 
  Age * Phase(Label) & 0.02 & 0.01 & 2.07 & 0.04 & * \\ 
  Age * Phase(Planning) & -0.00 & 0.01 & -0.31 & 0.76 &  \\ 
  Age * Phase(Response) & -0.00 & 0.01 & -0.09 & 0.93 &  \\ 
  Gaze * Phase(Label) & -0.06 & 0.10 & -0.59 & 0.56 &  \\ 
  Gaze * Phase(Planning) & -0.00 & 0.10 & -0.00 & 1 &  \\ 
  Gaze * Phase(Response) & 0.08 & 0.10 & 0.79 & 0.43 &  \\ 
  Trial type(FN) * Age * Gaze & 0.01 & 0.02 & 0.46 & 0.65 &  \\ 
  Trial type(NN) * Age * Gaze & 0.01 & 0.02 & 0.49 & 0.62 &  \\ 
  Trial type(FN) * Age * Phase(Label) & 0.01 & 0.02 & 0.73 & 0.47 &  \\ 
  Trial type(NN) * Age * Phase(Label) & -0.00 & 0.02 & -0.19 & 0.85 &  \\ 
  Trial type(FN) * Age * Phase(Planning) & -0.00 & 0.02 & -0.04 & 0.96 &  \\ 
  Trial type(NN) * Age * Phase(Planning) & 0.01 & 0.02 & 0.78 & 0.43 &  \\ 
  Trial type(FN) * Age * Phase(Response) & 0.02 & 0.02 & 1.14 & 0.25 &  \\ 
  Trial type(NN) * Age * Phase(Response) & 0.02 & 0.02 & 1.37 & 0.17 &  \\ 
  Trial type(FN) * Gaze * Phase(Label) & -0.05 & 0.15 & -0.32 & 0.75 &  \\ 
  Trial type(NN) * Gaze * Phase(Label) & 0.05 & 0.15 & 0.32 & 0.75 &  \\ 
  Trial type(FN) * Gaze * Phase(Planning) & -0.20 & 0.15 & -1.35 & 0.18 &  \\ 
  Trial type(NN) * Gaze * Phase(Planning) & -0.23 & 0.15 & -1.61 & 0.11 &  \\ 
  Trial type(FN) * Gaze * Phase(Response) & -0.43 & 0.15 & -2.97 & 0 & ** \\ 
  Trial type(NN) * Gaze * Phase(Response) & -0.14 & 0.15 & -0.96 & 0.34 &  \\ 
  Age * Gaze * Phase(Label) & -0.02 & 0.02 & -1.14 & 0.26 &  \\ 
  Age * Gaze * Phase(Planning) & 0.01 & 0.02 & 0.45 & 0.65 &  \\ 
  Age * Gaze * Phase(Response) & -0.00 & 0.02 & -0.18 & 0.86 &  \\ 
  Trial type(FN) * Age * Gaze * Phase(Label) & -0.01 & 0.02 & -0.41 & 0.68 &  \\ 
  Trial type(NN) * Age * Gaze * Phase(Label) & 0.01 & 0.02 & 0.47 & 0.64 &  \\ 
  Trial type(FN) * Age * Gaze * Phase(Planning) & -0.00 & 0.02 & -0.07 & 0.94 &  \\ 
  Trial type(NN) * Age * Gaze * Phase(Planning) & -0.02 & 0.02 & -0.64 & 0.52 &  \\ 
  Trial type(FN) * Age * Gaze * Phase(Response) & -0.04 & 0.02 & -1.64 & 0.1 &  \\ 
  Trial type(NN) * Age * Gaze * Phase(Response) & -0.05 & 0.02 & -1.94 & 0.05 & . \\ 
   \hline
\end{tabular}
\caption{Predictor estimates with standard errors and significance information for a linear mixed-effects model predicting social referencing in Experiment 1.} 
\label{tab:exp2_reg}
\end{table}

\subsubsection{Social referencing and
accuracy}\label{social-referencing-and-accuracy}

In the metacognitive framework, confidence judgments are generally
compared for correct and incorrect responses, with lower confidence for
incorrect responses taken as evidence for metacognitive accuracy
{[}Yeung \& Summerfield (2012). Here, a parallel approach would be to
examine the amount of social referencing children demonstrate for
correct and incorrect responses.

To this end, we fit a mixed-effects linear regression model with the
following structure:
\texttt{number\ of\ looks\ \textasciitilde{}\ accuracy\ *\ age\ in\ months\ *\ phase\ +\ (1\ \textbar{}\ subject)}.
The number of looks was collapsed across trial types and gaze
conditions, since there were relatively few incorrect trials overall and
none at all for some of the conditions. Results showed that accuracy and
phase interacted such that children looked at the experimenter
significantly more during the labeling phase of accurate trials
(\(\beta\) = 0.4, \emph{p} \textless{} .001), and significantly less in
the \emph{response} phase of accurate trials (\(\beta\) = -1.0, \emph{p}
\textless{} .001).

These findings confirm that children reference the experimenter more
while responding incorrectly. Importantly, this analysis was restricted
to trials in which there was a correct response available (i.e., because
the target was familiar or could be inferred by excluding the familiar
distracter or observing the speaker's gaze). Thus, children's looking is
sensitive not only to complete ignorance, but other factors that affect
accuracy, which might include graded evidence or relative familiarity
with the stimuli.

Additionally, children were more likely to answer correctly if they
referenced the speaker during the labeling phase. This pattern could be
due to children obtaining referential evidence from the experimenter's
gaze in the gaze condition, or it could represent trials in which
children were more engaged in the task.

\begin{CodeChunk}
\begin{figure*}[b]

{\centering \includegraphics[width=6in,height=3in]{figs/acc_results_e2-1} 

}

\caption[Results of Experiment 2]{Results of Experiment 2. Number of looks to the experimenter based on accuracy, phase, and age, collapsed across trial types and gaze conditions. Age in months was entered as a continuous variable in regression models but is presented here as a categorical variable for visual simplicity. Error bars are 95 percent confidence intervals. }\label{fig:acc_results_e2}
\end{figure*}
\end{CodeChunk}

\section{General Discussion}\label{general-discussion}

During the preschool years, children are increasingly able to actively
gather information through help-seeking and exploration (Chouinard et
al., 2007; Schulz \& Bonawitz, 2007). Do children monitor uncertainty in
their knowledge to guide these behaviors, or are they indiscriminate
with regard to underlying knowledge states? Here, we examined whether
young children's social referencing during a word-learning task was
driven by uncertainty about a label's referent.

We found that referential ambiguity strongly predicted children's social
referencing. Specifically, we observed this selectivity when children
were physically executing their decision, by placing the chosen object
in a bucket in Experiments 1 and 2 and also while reaching for an object
in Experiment 1. We speculate that children referenced the speaker
during the decision process because they expected evaluative feedback
about their demonstrated choice, either implicitly through the adult's
facial expressions, or through an explicit response. This idea is
consistent with other recent findings that preschoolers and toddlers
seek help selectively when a problem is difficult or they are less
skilled (Goupil et al., 2016; Vredenburgh \& Kushnir, 2015).

Intriguingly, we also found that in the aggregate children's looking
reflected graded referential evidence. In the case of FN trials,
children could solve the problem of reference by excluding the familiar
item (Markman \& Wachtel, 1988; Merriman, Bowman, \& MacWhinney, 1989),
and indeed, they chose the correct object most of the time for these
trials. If children simply monitored the presence or absence of such
cues, they would have consistently responded to FN trials with
certainty. Instead, their social referencing depended on a combination
of cues, suggesting that preschoolers are sensitive to graded
referential evidence. Meanwhile, children demonstrated uncertainty on
trials with only one cue to reference (i.e., FN trials with no
referential gaze and NN trials with referential gaze), suggesting that
they remain subjectively uncertain about new label-object mappings if
they have not received confirmation of its accuracy. These findings are
important given that being able to monitor the likelihood of accuracy
based on graded evidence is assumed important for decision-making and
behavioral regulation across development (Krebs \& Roebers, 2010; Yeung
\& Summerfield, 2012). For example, being able to monitor graded
epistemic uncertainty allows individuals to gate out information that
does not meet a criterion level of certainty based on individual goals
or task demands (Koriat \& Goldsmith, 1996).

These findings also bear on probabilistic models of learning and
cognition, which predict that the proportion of individuals endorsing a
belief will be proportional to the posterior probability of that belief
based on Bayesian inference (e.g., E. Bonawitz et al., 2014; Denison,
Bonawitz, Gopnik, \& Griffiths, 2013; E. Vul, Goodman, Griffiths, \&
Tenenbaum, 2014). The current findings are consistent with the
possibility that preschoolers might engage in this type of reasoning,
given that their uncertainty tracked quantitatively with the amount of
evidence available.

On the other hand, we found no evidence for selective social referencing
while the speaker was producing the label. One possibility is that young
children do not recognize the need for disambiguating information until
they need to make a decision (Markman, 1977). Another possibility is
that preschool-aged children spontaneously look at a speaker regardless
of ambiguity, and additional looking was not necessary or possible. This
latter possibility seems more credible, given that children looked at
the speaker at least once during labeling on most trials. Notably, Vaish
et al. observed selective referencing during labeling among infants.
Since infants in that study were holding one of the objects during
labeling, referencing the speaker would have required them to disengage
from that object, and may therefore have been more costly, promoting
selectivity. Future research with preschoolers that includes a greater
trade-off between attentional options would help to distinguish among
these possibilities.

Overall, these results provide evidence that preschool-aged children
monitor graded uncertainty in their mental representations. Furthermore,
they act on that uncertainty through spontaneous information-seeking.
These behaviors may underlie the rapid learning observed across domains
in early childhood.

\section{Acknowledgements}\label{acknowledgements}

We thank Veronica Cristiano for assisting with data collection. EH was
supported by a generous gift from Kinedu SAPI de CV.

\section{References}\label{references}

\setlength{\parindent}{-0.1in} \setlength{\leftskip}{0.125in} \noindent

\hypertarget{refs}{}
\hypertarget{ref-Balcomb2008}{}
Balcomb, F. K., \& Gerken, L. (2008). Three-year-old children can access
their own memory to guide responses on a visual matching task.
\emph{Developmental Science}, \emph{11}(5), 750--760.

\hypertarget{ref-Baldwin1991}{}
Baldwin, D. (1991). Infants' contribution to the achievement of joint
reference. \emph{Child Development}, \emph{62}(5), 875--890.

\hypertarget{ref-Baldwin1993}{}
Baldwin, D. (1993). Early referential understanding: infants' ability to
recognize referential acts for what they are. \emph{Developmental
Psychology}, \emph{29}, 832--843.

\hypertarget{ref-Bonawitz2014}{}
Bonawitz, E., Denison, S., Gopnik, A., \& Griffiths, T. L. (2014).
Win-Stay, Lose-Sample: A simple sequential algorithm for approximating
Bayesian inference. \emph{Cognitive Psychology}, \emph{74}(C), 35--65.

\hypertarget{ref-Call2001}{}
Call, J., \& Carpenter, M. (2001). Do apes and children know what they
have seen? \emph{Animal Cognition}, \emph{3}(4), 207--220.

\hypertarget{ref-Chouinard2007}{}
Chouinard, M. M., Harris, P. L., \& Maratsos, M. P. (2007). Children's
questions: A mechanism for cognitive development. \emph{Monographs of
the Society for Research in Child Development}, \emph{72}, 1--129.

\hypertarget{ref-Corriveau2009}{}
Corriveau, K., \& Harris, P. L. (2009). Preschoolers continue to trust a
more accurate informant 1 week after exposure to accuracy information.
\emph{Developmental Science}, \emph{12}(1), 188--193.

\hypertarget{ref-Coughlin2014}{}
Coughlin, C., Hembacher, E., Lyons, K. E., \& Ghetti, S. (2014).
Introspection on uncertainty and judicious help-seeking during the
preschool years. \emph{Developmental Science}, \emph{18}(6), 957--971.

\hypertarget{ref-Csibra2006}{}
Csibra, G., \& Gergely, G. (2006). Social learning and social cognition:
The case for pedagogy. In Y. Munakata \& M. H. Johnson (Eds.),
\emph{Processes of change in brain and cognitive development} (pp.
249--274). Oxford: Oxford University Press: Oxford University Press.

\hypertarget{ref-Daw2007}{}
Daw, N. D., \& Courville, A. C. (2007). The pigeon as particle filter.
In \emph{Advances in neural information processing systems}.

\hypertarget{ref-Denison2013}{}
Denison, S., Bonawitz, E., Gopnik, A., \& Griffiths, T. L. (2013).
Rational variability in children's causal inferences: The Sampling
Hypothesis. \emph{Cognition}, \emph{126}(2), 285--300.

\hypertarget{ref-Flavell1995}{}
Flavell, J. H., Green, F. L., Flavell, E. R., Harris, P. L., \&
Astington, J. W. (1995). Young children's knowledge about thinking.
\emph{Monographs of the Society for Research in Child Development},
i--iii--v--vi--1--113.

\hypertarget{ref-Ghetti2013}{}
Ghetti, S., Hembacher, E., \& Coughlin, C. (2013). Feeling Uncertain and
Acting on It During the Preschool Years: A Metacognitive Approach.
\emph{Child Development Perspectives}, \emph{7}(3), 160--165.

\hypertarget{ref-Goupil2016}{}
Goupil, L., Romand-Monnier, M., \& Kouider, S. (2016). Infants ask for
help when they know they dont know. \emph{Proceedings of the National
Academy of Sciences}, \emph{113}(13), 3492--3496.

\hypertarget{ref-Gureckis2012}{}
Gureckis, T. M., \& Markant, D. B. (2012). Self-Directed Learning.
\emph{Perspectives on Psychological Science}, \emph{7}(5), 464--481.

\hypertarget{ref-Hembacher2014}{}
Hembacher, E., \& Ghetti, S. (2014). Dont look at my answer: Subjective
uncertainty underlies preschoolers exclusion of their least accurate
memories. \emph{Psychological Science}, \emph{25}(9),
0956797614542273--1776.

\hypertarget{ref-Jaswal2010}{}
Jaswal, V. K., Croft, A. C., Setia, A. R., \& Cole, C. A. (2010). Young
Children Have a Specific, Highly Robust Bias to Trust Testimony.
\emph{Psychological Science}, \emph{21}(10), 1541--1547.

\hypertarget{ref-Kidd2012}{}
Kidd, C., Piantadosi, S. T., \& Aslin, R. N. (2012). The Goldilocks
Effect: Human Infants Allocate Attention to Visual Sequences That Are
Neither Too Simple Nor Too Complex. \emph{PLoS ONE}, \emph{7}(5),
e36399--8.

\hypertarget{ref-Kidd2014}{}
Kidd, C., Piantadosi, S. T., \& Aslin, R. N. (2014). The Goldilocks
Effect in Infant Auditory Attention. \emph{Child Development},
\emph{19}, n/a--n/a.

\hypertarget{ref-Koenig2009}{}
Koenig, M. A., Clement, F., \& Harris, P. L. (2009). Trust in testimony:
Children's use of true and false statements. \emph{Psychological
Science}, \emph{15}, 694--698.

\hypertarget{ref-Koriat1996}{}
Koriat, A., \& Goldsmith, M. (1996). Monitoring and control processes in
the strategic regulation of memory accuracy. \emph{Psychological
Review}, \emph{103}(3), 490--517.

\hypertarget{ref-Krebs2010}{}
Krebs, S. S., \& Roebers, C. M. (2010). Children's strategic regulation,
metacognitive monitoring, and control processes during test taking.
\emph{British Journal of Educational Psychology}, \emph{80}(3),
325--340.

\hypertarget{ref-Li2014}{}
Li, Q.-G., Heyman, G. D., Xu, F., \& Lee, K. (2014). Young children's
use of honesty as a basis for selective trust. \emph{Journal of
Experimental Child Psychology}, \emph{117}(C), 59--72.

\hypertarget{ref-Lyons2011a}{}
Lyons, K. E., \& Ghetti, S. (2011). The development of uncertainty
monitoring in early childhood. \emph{Child Development}, \emph{82}(6),
1778--1787.

\hypertarget{ref-Lyons2011}{}
Lyons, K. E., \& Zelazo, P. D. (2011). Monitoring, metacognition, and
executive function: Elucidating the role of self-reflection in the
development of self-regulation. \emph{Advances in Child Development and
Behavior}, \emph{40}, 379--412.

\hypertarget{ref-Markman1977}{}
Markman, E. M. (1977). Realizing that you don't understand: A
preliminary investigation. \emph{Child Development}, \emph{48}(3),
986--992.

\hypertarget{ref-Markman1988}{}
Markman, E. M., \& Wachtel, G. F. (1988). Children's use of mutual
exclusivity to constrain the meanings of words. \emph{Cognitive
Psychology}, \emph{20}, 121--157.

\hypertarget{ref-Merriman1989}{}
Merriman, W. E., Bowman, L. L., \& MacWhinney, B. (1989). The mutual
exclusivity bias in children's word learning. \emph{Monographs of the
Society for Research in Child Development}, \emph{54},
i--iii--v--1--129.

\hypertarget{ref-Ruggeri2017}{}
Ruggeri, A., Sim, Z. L., \& Xu, F. (2017). ``Why is Toma late to school
again?'' Preschoolers identify the most informative questions.
\emph{Developmental Science}.

\hypertarget{ref-Sabbagh2001}{}
Sabbagh, M. A., \& Baldwin, D. (2001). Learning words from knowledgeable
versus ignorant speakers: Links between preschoolers theory of mind and
semantic development. \emph{Child Development}, \emph{72}, 1054--1070.

\hypertarget{ref-Schneider1998}{}
Schneider, W. (1998). Performance prediction in young children: Effects
of skill, metacognition and wishful thinking. \emph{Developmental
Science}, 291--297.

\hypertarget{ref-Schulz2007}{}
Schulz, L. E., \& Bonawitz, E. B. (2007). Serious fun: Preschoolers
engage in more exploratory play when evidence is confounded.
\emph{Developmental Psychology}, \emph{43}(4), 1045--1050.

\hypertarget{ref-Sim2017}{}
Sim, Z. L., \& Xu, F. (2017). Learning higher-order generalizations
through free play: Evidence from two- and three-year-old children.
\emph{Developmental Psychology}.

\hypertarget{ref-Sobel2013}{}
Sobel, D. M., \& Kushnir, T. (2013). Knowledge matters: How children
evaluate the reliability of testimony as a process of rational
inference. \emph{Psychological Review}, \emph{120}, 779=797.

\hypertarget{ref-Sodian2012}{}
Sodian, B., Thoermer, C., Kristen, S., \& Perst, H. (2012).
Metacognition in infants and young children. In M. J. Beran, J. Brandl,
J. Perner, \& J. Proust (Eds.), \emph{Foundations of metacognition} (pp.
119--133).

\hypertarget{ref-Sorce1985}{}
Sorce, J. F., Emde, R. N., Campos, J., \& Klinnert, M. D. (1985).
Maternal emotional signaling: Its effect on the visual cliff behavior of
1-year-olds. \emph{Developmental Psychology}, \emph{21}, 195--200.

\hypertarget{ref-Stahl2015}{}
Stahl, A. E., \& Feigenson, L. (2015). Observing the unexpected enhances
infants learning and exploration. \emph{Science}, \emph{348}(6230),
91--94.

\hypertarget{ref-Striano2000}{}
Striano, T., \& Rochat, P. (2000). Emergence of selective social
referencing in infancy. \emph{Infancy}, \emph{1}, 253--264.

\hypertarget{ref-TamisLeMonda2008}{}
Tamis LeMonda, C. S., Adolph, K. E., Lobo, S. A., Karasik, L. B., Ishak,
S., \& Dimitropoulou, K. A. (2008). When infants take mothers' advice:
18-month-olds integrate perceptual and social information to guide motor
action. \emph{Developmental Psychology}, \emph{44}(3), 734--746.

\hypertarget{ref-Teglas2011}{}
Teglas, E., Vul, E., Girotto, V., Gonzalez, M., Tenenbaum, J. B., \&
Bonatti, L. L. (2011). Pure Reasoning in 12-Month-Old Infants as
Probabilistic Inference. \emph{Science}, \emph{332}(6033), 1054--1059.

\hypertarget{ref-Tomasello1986}{}
Tomasello, M., \& Farrar, M. J. (1986). Joint attention and early
language. \emph{Child Development}, \emph{57}, 1454--1463.

\hypertarget{ref-Vaish2011}{}
Vaish, A., Demir, Ö. E., \& Baldwin, D. (2011). Thirteen- and
18-month-old infants recognize when they need referential information.
\emph{Social Development}, \emph{20}(3), 431--449.

\hypertarget{ref-Vredenburgh2015}{}
Vredenburgh, C., \& Kushnir, T. (2015). Young children's help-seeking as
active information gathering. \emph{Cognitive Science}, \emph{40}(3),
697--722.

\hypertarget{ref-Vul2014}{}
Vul, E., Goodman, N., Griffiths, T. L., \& Tenenbaum, J. B. (2014). One
and Done? Optimal Decisions From Very Few Samples. \emph{Cognitive
Science}, \emph{38}(4), 599--637.

\hypertarget{ref-Walden1988}{}
Walden, T. A., \& Ogan, T. A. (1988). The development of social
referencing. \emph{Child Development}, \emph{59}(5), 1230--1240.

\hypertarget{ref-Yeung2012}{}
Yeung, N., \& Summerfield, C. (2012). Metacognition in human
decision-making: confidence and error monitoring. \emph{Philosophical
Transactions of the Royal Society B: Biological Sciences},
\emph{367}(1594), 1310--1321.

\hypertarget{ref-Zarbatany1985}{}
Zarbatany, L., \& Lamb, M. E. (1985). Social referencing as a function
of information source: Mothers versus strangers. \emph{Infant Behavior
and Development}, \emph{8}, 25--33.

\bibliography{}

\end{document}
